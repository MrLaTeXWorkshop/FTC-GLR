\begin{question}

 $\int$ : R* $\rightarrow$ R tal que f(x) = 2x
\begin{figure}[ht]
\centering
\resizebox{.2\columnwidth}{!}{%
\begin{tikzpicture}[
  every node/.style={on grid},
  setA/.style={fill=blue,circle,inner sep=2pt},
  setC/.style={fill=red,rectangle,inner sep=2pt},
  every fit/.style={draw,fill=blue!15,ellipse,text width=25pt},
  >=latex
]

% set A
\node[setA,label=left:$4$] (a) {};
\node [setA,below = of a,label=left:$2$] (b) {};
\node[above=of a,anchor=south] {$A$};

% set B
\node[inner sep=0pt,right=3cm of a] (x) {$4$};
\node[below = of x] (y) {$2$};
\node[inner sep=0pt,below = of y] (z) {$0$};
\node[above=of x,anchor=south] {$B$};

% the arrows
\draw[->,shorten >= 3pt] (a) -- node[label=above:$f$] {} (x);
\draw[->,shorten >= 3pt] (b) -- node[label=above:$f$] {} (y);

% the boxes around the sets
\begin{pgfonlayer}{background}
\node[fit= (a)  (b) ] {};
\node[fit= (x) (z) ] {};
\end{pgfonlayer}
\end{tikzpicture}
}
\caption{f é total e injetora, mas não sobrejetora.}
\end{figure}

$\int$ : R $\rightarrow$ R+ tal que f(x) = x$^{2}$
\begin{figure}[ht]
\centering
\resizebox{.2\columnwidth}{!}{%
\begin{tikzpicture}[
  every node/.style={on grid},
  setA/.style={fill=blue,circle,inner sep=2pt},
  setC/.style={fill=red,rectangle,inner sep=2pt},
  every fit/.style={draw,fill=blue!15,ellipse,text width=25pt},
  >=latex
]

% set A
\node[setA,label=left:$2$] (a) {};
\node [setA,below = of a,label=left:$-2$] (b) {};
\node[above=of a,anchor=south] {$A$};

% set B
\node[inner sep=0pt,right=3cm of a] (x) {$4$};
\node[above=of x,anchor=south] {$B$};

% the arrows
\draw[->,shorten >= 3pt] (a) -- node[label=above:$f$] {} (x);
\draw[->,shorten >= 3pt] (b) -- node[label=above:$f$] {} (x);

% the boxes around the sets
\begin{pgfonlayer}{background}
\node[fit= (a)  (b) ] {};
\node[fit= (x) (z) ] {};
\end{pgfonlayer}
\end{tikzpicture}
}
\caption{f é total e sobrejetora, mas não injetora.}
\end{figure}

 $\int$ : N $\rightarrow$ N tal que f(x) = 2x - 1
\begin{figure}[ht]
\centering
\resizebox{.2\columnwidth}{!}{%
\begin{tikzpicture}[
  every node/.style={on grid},
  setA/.style={fill=blue,circle,inner sep=2pt},
  setC/.style={fill=red,rectangle,inner sep=2pt},
  every fit/.style={draw,fill=blue!15,ellipse,text width=25pt},
  >=latex
]

% set A
\node[setA,label=left:$2$] (a) {};
\node [setA,below = of a,label=left:$1$] (b) {};
\node [setA,below = of b,label=left:$0$] (c) {};
\node[above=of a,anchor=south] {$A$};

% set B
\node[inner sep=0pt,right=3cm of a] (x) {$3$};
\node[below = of x] (y) {$1$};
\node[above=of x,anchor=south] {$B$};

% the arrows
\draw[->,shorten >= 3pt] (a) -- node[label=above:$f$] {} (x);
\draw[->,shorten >= 3pt] (b) -- node[label=above:$f$] {} (y);

% the boxes around the sets
\begin{pgfonlayer}{background}
\node[fit= (a)  (c) ] {};
\node[fit= (x) (y) ] {};
\end{pgfonlayer}
\end{tikzpicture}
}
\caption{f não é total, mas é injetora e sobrejetora.}
\end{figure}
\end{question}

\newpage